%---------------------------------------------------------------------
%
%                        TeXiS_indice.tex
%
%---------------------------------------------------------------------
%
% TeXiS_indice.tex
% Copyright 2009 Marco Antonio Gomez-Martin, Pedro Pablo Gomez-Martin
%
% This file belongs to TeXiS, a LaTeX template for writting
% Thesis and other documents. The complete last TeXiS package can
% be obtained from http://gaia.fdi.ucm.es/projects/texis/
%
% This work may be distributed and/or modified under the
% conditions of the LaTeX Project Public License, either version 1.3
% of this license or (at your option) any later version.
% The latest version of this license is in
%   http://www.latex-project.org/lppl.txt
% and version 1.3 or later is part of all distributions of LaTeX
% version 2005/12/01 or later.
%
% This work has the LPPL maintenance status `maintained'.
% 
% The Current Maintainers of this work are Marco Antonio Gomez-Martin
% and Pedro Pablo Gomez-Martin
%
%---------------------------------------------------------------------
%
% Contiene  los  comandos  para  generar  el �ndice  de  palabras  del
% documento.
%
%---------------------------------------------------------------------
%
% NOTA IMPORTANTE: el  soporte en TeXiS para el  �ndice de palabras es
% embrionario, y  de hecho  ni siquiera se  describe en el  manual. Se
% proporciona  una infraestructura  b�sica (sin  terminar)  para ello,
% pero  no ha  sido usada  "en producci�n".  De hecho,  a pesar  de la
% existencia de  este fichero, *no* se incluye  en Tesis.tex. Consulta
% la documentaci�n en TeXiS_pream.tex para m�s informaci�n.
%
%---------------------------------------------------------------------

% Nombre Indice Palabras
\def\indicePalabrasVal{Index}
\newcommand{\indicePalabras}[1]{
\def\indicePalabrasVal{#1}
}

% Nombre Indice Citas
\def\indiceCitasVal{Index}
\newcommand{\indiceCitas}[1]{
\def\indiceCitasVal{#1}
}

% Para pedir que se genere  la informaci�n para el �ndice de palabras.
% Para  no  ralentizar la  compilaci�n,  s�lo  se  genera cuando  est�
% definido el s�mbolo correspondiente.

\ifx\generaindice\undefined
\else
\makeindex
\fi

% Pruebas con �ndices
\usepackage{index}
\newindex{default}{idx}{ind}{\indicePalabrasVal}
%\newindex{default}{idx}{ind}{�ndice}
\newindex{cite}{cdx}{cnd}{\indiceCitasVal}
%\newindex{cite}{cdx}{cnd}{�ndice de citas}
\renewcommand{\citeindextype}{cite}
\citeindextrue


% Si se  va a generar  la tabla de  contenidos (el �ndice  habitual) y
% tambi�n vamos a  generar el �ndice de palabras  (ambas decisiones se
% toman en  funci�n de  la definici�n  o no de  un par  de constantes,
% puedes consultar modo.tex para m�s informaci�n), entonces metemos en
% la tabla de contenidos una  entrada para marcar la p�gina donde est�
% el �ndice de palabras.


% Ponemos el marcador en el PDF al nivel adecuado, dependiendo
% de su hubo partes en el documento o no (si las hay, queremos
% que aparezca "al mismo nivel" que las partes.
%
%\ifx\generatoc\undefined
%\ifx\generaindice\undefined
%\else
%	\ifx\tienePartesTeXiS\undefined
%   	\addcontentsline{toc}{chapter}{\indexname}
%   	\addcontentsline{toc}{chapter}{\indicePalabrasVal}
%   	\addcontentsline{toc}{chapter}{\indiceCitasVal}
%	   \pdfbookmark[0]{\indexname}{\indexname}
%	   \pdfbookmark[0]{\indicePalabrasVal}{\indicePalabrasVal}
%	   \pdfbookmark[0]{\indiceCitasVal}{\indiceCitasVal}
%	\else
%   	\addcontentsline{toc}{part}{\indexname}
%   	\addcontentsline{toc}{part}{\indicePalabrasVal}
%   	\addcontentsline{toc}{part}{\indiceCitasVal}
%	   \pdfbookmark[-1]{\indexname}{\indexname}
%	   \pdfbookmark[-1]{\indicePalabrasVal}{\indicePalabrasVal}
%	   \pdfbookmark[-1]{\indiceCitasVal}{\indiceCitasVal}
%	\fi
%\fi



% Variable local para emacs, para  que encuentre el fichero maestro de
% compilaci�n y funcionen mejor algunas teclas r�pidas de AucTeX

%%%
%%% Local Variables:
%%% mode: latex
%%% TeX-master: "./tesis.tex"
%%% End:
