%---------------------------------------------------------------------
%
%                          Bibliography
%
%---------------------------------------------------------------------
%
% bibliografia.tex
% Copyright 2015 Dr. Francisco J. Pulido
%
% This file belongs to the PhD titled "New Techniques and Algorithms for Multiobjective and Lexicographic Goal-Based Shortest Path Problems", distributed under the Creative Commons Licence Attribution-NonCommercial-NoDerivs 3.0, available in http://creativecommons.org/licenses/by-nc-nd/3.0/. The complete PhD dissertation is freely accessible from http://www.lcc.uma.es/~francis/
%
% This thesis has been written adapting the TeXiS template, a LaTeX template for writting thesis and other documents. The complete TeXiS package can be obtained from http://gaia.fdi.ucm.es/projects/texis/. TeXis is distributed under the same conditions of the LaTeX Project Public License (http://www.latex-project.org/lppl.txt). The complete license is available in http://creativecommons.org/licenses/by-sa/3.0/legalcode
%
%---------------------------------------------------------------------

%---------------------------------------------------------------------
%
% Fichero  que  configura  los  par�metros  de  la  generaci�n  de  la
% bibliograf�a.  Existen dos  par�metros configurables:  los ficheros
% .bib que se utilizan y la frase c�lebre que aparece justo antes de la
% primera referencia.
%
%---------------------------------------------------------------------

%%%%%%%%%%%%%%%%%%%%%%%%%%%%%%%%%%%%%%%%%%%%%%%%%%%%%%%%%%%%%%%%%%%%%%
% Definici�n de los ficheros .bib utilizados:
% \setBibFiles{<lista ficheros sin extension, separados por comas>}
% Nota:
% Es IMPORTANTE que los ficheros est�n en la misma l�nea que
% el comando \setBibFiles. Si se desea utilizar varias l�neas,
% terminarlas con una apertura de comentario.
%%%%%%%%%%%%%%%%%%%%%%%%%%%%%%%%%%%%%%%%%%%%%%%%%%%%%%%%%%%%%%%%%%%%%%
\setBibFiles{%
Tesis%
}

%%%%%%%%%%%%%%%%%%%%%%%%%%%%%%%%%%%%%%%%%%%%%%%%%%%%%%%%%%%%%%%%%%%%%%
% Definici�n de la frase c�lebre para el cap�tulo de la
% bibliograf�a. Dentro normalmente se querr� hacer uso del entorno
% \begin{FraseCelebre}, que contendr� a su vez otros dos entornos,
% un \begin{Frase} y un \begin{Fuente}.
%
% Nota:
% Si no se quiere cita, se puede eliminar su definici�n (en la
% macro setCitaBibliografia{} ).
%%%%%%%%%%%%%%%%%%%%%%%%%%%%%%%%%%%%%%%%%%%%%%%%%%%%%%%%%%%%%%%%%%%%%%
%\setCitaBibliografia{
%\begin{FraseCelebre}
%\begin{Frase}
%
%\end{Frase}
%\begin{Fuente}
%
%\end{Fuente}
%\end{FraseCelebre}
%}

%%
%% Creamos la bibliografia
%%
%\makeBib

\def\titBib{Bibliography}

\makeBibTopic


