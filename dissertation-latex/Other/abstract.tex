%---------------------------------------------------------------------
%
%                          Abstract
%
%---------------------------------------------------------------------
%
% abstract.tex
% Copyright 2015 Dr. Francisco J. Pulido
%
% This file belongs to the PhD titled "New Techniques and Algorithms for Multiobjective and Lexicographic Goal-Based Shortest Path Problems", distributed under the Creative Commons Licence Attribution-NonCommercial-NoDerivs 3.0, available in http://creativecommons.org/licenses/by-nc-nd/3.0/. The complete PhD dissertation is freely accessible from http://www.lcc.uma.es/~francis/
%
% This thesis has been written adapting the TeXiS template, a LaTeX template for writting thesis and other documents. The complete TeXiS package can be obtained from http://gaia.fdi.ucm.es/projects/texis/. TeXis is distributed under the same conditions of the LaTeX Project Public License (http://www.latex-project.org/lppl.txt). The complete license is available in http://creativecommons.org/licenses/by-sa/3.0/legalcode
%
%---------------------------------------------------------------------

\chapter*{Abstract}
\cabeceraEspecial{Abstract}

Shortest Path Problems (SPP) are one of the most extensively studied problems in the fields of Artificial Intelligence (AI) and Operations Research (OR). It consists in finding the shortest path between two given nodes in a graph such that the sum of the weights of its constituent arcs is minimized. However, real life problems frequently involve the consideration of multiple, and often conflicting, criteria. When multiple objectives must be simultaneously optimized, the concept of a single optimal solution is no longer valid. Instead, a set of efficient or Pareto-optimal solutions define the optimal trade-off between the objectives under consideration.  

The Multicriteria Search Problem (MSP), or Multiobjective Shortest Path Problem, is the natural extension to the SPP when more than one criterion are considered. The MSP is computationally harder than the single objective one. The number of label expansions can grow exponentially with solution depth, even for the two objective case \citep{hansen1979}. However, with the assumption of bounded integer costs and a fixed number of objectives the problem becomes tractable for polynomially sized graphs (e.g. see \citep{Mandow2009,Muller-Hannemann2006}).

A wide variety of practical application in different fields can be identified for the MSP, like robot path planning \citep{Wu2011}, hazardous material transportation \citep{caramiaetal2010}, route planning  \citep{Jozefowiez2008}, optimization of public transportation \citep{Raith2009}, QoS in networks \citep{craveirinha2009}, or routing in multimedia networks \citep{climacoetal2003}.

Goal programming is one of the most successful Multicriteria Decision Making (MCDM) techniques used in Multicriteria Optimization. In this thesis we explore one of its variants in the MSP. Thus, we aim to solve the Multicriteria Search Problem with lexicographic goal-based preferences. To do so, we build on previous work on algorithm \namoa, a successful extension of the \astar \ algorithm to the multiobjective case. More precisely, we provide a new algorithm called \lexgo, an exact label-setting algorithm that returns the subset of Pareto optimal paths that satisfy a set of lexicographic goals, or the subset that minimizes deviation from goals if these cannot be fully satisfied. Moreover, \lexgo \ is proved to be admissible and expands only a subset of the labels expanded by an optimal algorithm like \namoa, which performs a full Multiobjective Search.

Since time rather than memory is the limiting factor in the performance of  multicriteria search algorithms, we also propose a new technique called \emph{t-discarding} to speed up dominance checks in the process of discarding new alternatives during the search. The application of \emph{t-discarding} to the algorithms studied previously, \namoa \ and \lexgo, leads to the introduction of two new time-efficient algorithms named \namoate \ and \lexgote, respectively.

All the algorithmic alternatives are tested in two scenarios, random grids and realistic road maps problems. The experimental evaluation shows the effectiveness of \lexgo \ in both benchmarks, as well as the dramatic reductions of time requirements experienced by the t-discarding versions of the algorithms, with respect to the ones with traditional pruning.

\endinput

