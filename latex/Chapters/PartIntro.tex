%---------------------------------------------------------------------
%
%                          Part Intro
%
%---------------------------------------------------------------------
%
% PartIntro.tex
% Copyright 2015 Dr. Francisco J. Pulido
%
% This file belongs to the PhD titled "New Techniques and Algorithms for Multiobjective and Lexicographic Goal-Based Shortest Path Problems", distributed under the Creative Commons Licence Attribution-NonCommercial-NoDerivs 3.0, available in http://creativecommons.org/licenses/by-nc-nd/3.0/. The complete PhD dissertation is freely accessible from http://www.lcc.uma.es/~francis/
%
% This thesis has been written adapting the TeXiS template, a LaTeX template for writting thesis and other documents. The complete TeXiS package can be obtained from http://gaia.fdi.ucm.es/projects/texis/. TeXis is distributed under the same conditions of the LaTeX Project Public License (http://www.latex-project.org/lppl.txt). The complete license is available in http://creativecommons.org/licenses/by-sa/3.0/legalcode
%
%---------------------------------------------------------------------

\partTitle{Motivation and Fundamentals}

\partDesc{This first part of this thesis is divided into three chapters. The first one is devoted to introduce its field of study. We enumerate the goals of this dissertation and introduce the contributions that we will develop further in the second part. The second chapter describes our main subject of study and research, Multicriteria Search problems and algorithms. The third chapter is dedicated to the benchmarks used to conduct the experimental evaluation. Thus, in particular: 

\begin{itemize}
    \item Chapter \ref{chapIntroduction} gives an overview of the motivation, scope and goals of this thesis, and enumerates its contributions.  
    \item Chapter \ref{chapMultiObjAlg} defines the Multicriteria Search Problem, gives some examples of application and classifies the approaches to deal with the problem. Formal properties of the lower bounds to apply to the multicriteria search algorithms are also described. Finally, \namoa \ is introduced emphasizing its relevant features and properties to our own work.
    \item Chapter \ref{chapMultiObjTestBeds} enumerates relevant benchmarks employed in the literature and describes the test beds used in this thesis, as well as the main parameters followed in the experimental evaluation.
\end{itemize}
}

\makepart
