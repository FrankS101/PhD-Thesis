%---------------------------------------------------------------------
%
%                      TeXiS_certif.tex
%
%---------------------------------------------------------------------
%
% Contiene el cap�tulo de la certificacion.
%
% Se crea como un cap�tulo sin numeraci�n.
%
%---------------------------------------------------------------------

%%%
%% COMANDO PARA CREAR LA CERTIFICACI�N.
%% CONTIENE TODO EL C�DIGO LaTeX
%%%
\newcommand{\makeCertif}{


% Ponemos el marcador en el PDF
%\ifpdf
%\pdfbookmark{Certificaci�n}{certificacion}
%\fi


%%
%% Creamos la certificacion
%%

\thispagestyle{empty}\mbox{}

El Dr. D. \directorPortadaVal, Profesor Titular de Universidad, del �rea de \areaVal~ de la \facultadVal~ de la \institucionVal, 

\vspace{1.5cm} Certifica que, \vspace{1.5cm}

D. \autorPortadaVal, \titulacionVal, ha realizado en
el \departamentoVal~ de la \institucionVal, bajo su direcci�n, el trabajo de
investigaci�n correspondiente a su Tesis Doctoral titulada: \vspace{1.5cm}

\begin{center}
\emph{\tituloPortadaVal}
\end{center}

\vspace{1.5cm} Revisado el presente trabajo, estima que puede ser
presentado al tribunal que ha de juzgarlo, y autoriza la
presentaci�n de esta Tesis Doctoral en la \institucionVal.

\vspace{4cm}
\begin{tabular}{ccc}
  % after \\: \hline or \cline{col1-col2} \cline{col3-col4} ...
  Fdo.: Dr. \directorPortadaVal \\  %& \hspace{2cm} & Fdo.: \codirectorVal \\
\end{tabular}
\vspace{1.5cm}
\begin{flushright}
\lugarPublicacionVal, \fechaPublicacionVal
\end{flushright}

\newpage
\thispagestyle{empty}\mbox{}
%\endinput

} % \newcommand{\makeCertif}

% Variable local para emacs, para  que encuentre el fichero maestro de
% compilaci�n y funcionen mejor algunas teclas r�pidas de AucTeX
%%%
%%% Local Variables:
%%% mode: latex
%%% TeX-master: "../Tesis.tex"
%%% End:
