%---------------------------------------------------------------------
%
%                          conclusions.tex
%
%---------------------------------------------------------------------
%
% conclusions.tex
% Copyright 2015 Dr. Francisco J. Pulido
%
% This presentation belongs to the PhD titled "New Techniques and Algorithms for Multiobjective and Lexicographic Goal-Based Shortest Path Problems", distributed under the Creative Commons Licence Attribution-NonCommercial-NoDerivs 3.0, available in http://creativecommons.org/licenses/by-nc-nd/3.0/. The complete PhD dissertation is freely accessible from http://www.lcc.uma.es/~francis/

\section{Conclusions \& future work}

\subsection{Conclusions}
\begin{frame} 
\frametitle{Conclusions}
	\begin{enumerate}
		\item We have tackled the \textcolor{ao}{MSP with lexicographic preferences} from two approaches: \textcolor{ao}{a priori} and \textcolor{ao}{a posteriori}.
		\item On the a priori approach, we have contributed two new algorithms: \textcolor{ao}{\lexgo} \ and \textcolor{ao}{\lexgodr}. 
		\item On the a posteriori approach: \textcolor{ao}{\namoadr} \ based on \namoa.
		\item All algorithmic approaches have been \textcolor{ao}{formally proved to be admissible}.
		\item The contributions of this thesis \textcolor{ao}{outperform the state of the art} in \textcolor{red}{an order of magnitude} for medium size problems and \textcolor{red}{two orders of magnitude} for the hardest problems. 
		\item \textcolor{ao}{\namoadr} \ \textcolor{ao}{\lexgodr} represent \textcolor{ao}{the new state of the art} in \textcolor{ao}{MSP with lexicographic goals}.
	\end{enumerate}
\note{}
\end{frame}

\subsection{Future work}
\begin{frame} 
\frametitle{Future work}
	\begin{enumerate}
		\item Investigate more \textcolor{ao}{formal proofs about \lexgo}, concerning the optimality in its class of algorithms and the expansion of labels when using more informed lower bounds.
		\vspace{3mm}
		\item Extend \lexgo \ to be used with \textcolor{ao}{other GP models} or \textcolor{ao}{other formulas to measure the deviation} from goals. 
		\vspace{3mm}
		\item \textcolor{ao}{Research the applicability} of t-discarding to other multiobjective or multicriteria search algorithms with lower bounds.
		%\item Tighten the concept of dominance for those cases where only a small amount of solutions are required.
		\vspace{3mm}
		\item Bring multicriteria search algorithms to \textcolor{ao}{real route planning applications.} 
	\end{enumerate}
\note{}
\end{frame}
